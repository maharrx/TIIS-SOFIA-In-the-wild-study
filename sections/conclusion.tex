\section{Conclusion}
   
    This paper presents insight into the experiences of people living with \ac{AD} of a speech-enabled \ac{CA} for the self-report of mental health and wellbeing in an at-home setting over a period of 4 weeks. Results from thematic analysis of post-study interviews indicate that \ac{CA}s' conversational features have the capacity to support engaging and sustainable self-report experiences, revealing users' experiences as strongly shaped by strategies adopted to overcome \ac{CA}s' technical limitations, diverse personified perceptions of the agent, the socially-contingent nature of self-reporting practice, and users' reflections on privacy and security concerns.
    
    Participants' positive responses to the \ac{UEQ} and high levels of engagement with a \ac{CA} despite technical limitations furthermore serve as initial evidence of users' willingness to accept the use of these systems for the self-report of mental health and wellbeing in the home-context. Based on these findings, we discuss implications for the design of \ac{CA}s for mental health and wellbeing with respect to the challenges surfaced by this work including challenges of conversational pattern matching, filling unmet interpersonal gaps, and the use of self-report \ac{CA}s in the at-home social context.

% RQ1
% Our findings indicate users' positive perceptions of a \ac{CA} for the self-report of mental health and wellbeing as reflected in the actions taken to overcome the technological limitations of \ac{CA}s, diverse personified perceptions of the agent and its acceptance among users' social circles. 
% RQ2
% Participants expressed their social context and privacy concerns, including eavesdropping and data security, as the primary factors impacting their self-report experience. 
% RQ3
% The \ac{CA}'s conversational skill and the provision of space for self-reflection were considered important for the realisation of engaging and sustainable self-report experiences.
% 
% Based on our findings, we discuss the challenges of deploying these systems in practice, and means of navigating by design the opportunities these systems pose.
% ways to make use of the opportunities these systems pose to support mental health and wellbeing.
% How they perceive
% This study provides initial insights into the perceptions and behaviors of people living with \acf{AD} following four weeks of \ac{CA} use for the self-report of mental health and well-being. Results suggest that \ac{CA}s can serve as an engaging tool for emotional expression, despite outstanding technical limitations, social challenges and privacy concerns. Thematic analysis of participants' comments provided during post-study interviews revealed that users' perceptions of the agent as a good listener, companion, and tool for emotional venting and self-talking enabled them to express their emotions more naturally, made them feel heard, and facilitated positive relationships with their loved ones. Participants' high engagement and positive responses to the \acl{UEQ} scale suggest users' positive perception and acceptance of \ac{CA} for mental health self-reports.
% % 
% % In addition to the technical limitations preventing users from completing self-reports, we find that users' social context, and privacy challenges including eavesdropping and data security concerns were the primary barriers to mental health self-reports via \ac{CA}.
% % users' tactics
% % To overcome technical limitations and fully express their mental state, participants applied strategies including making multiple entries, adapting their speech and preparing their reflections in advance.
% % 
% % design suggestions
% Based on their self-reporting experiences, participants suggested improving the conversation design by varying questions more often, probing, and guiding the conversation as well as suggesting ways to reflect on their data to support engaging and sustainable mental health self-reports via \ac{CA}.
% % discussion
% Reflecting on our findings, we discuss the benefits and risks of employing \ac{CA}s for mental health self-report and implications for designing such \ac{CA}s, which we hope will guide the future design of \ac{CA}s for mental health and wellbeing.