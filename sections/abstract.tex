% MAX WORDS 400 
% Self-reports obtained by health questionnaires and diary logs are considered effective methods of monitoring and assessing affective disorders (ADs) including depression and bipolar disorder. While long-established, these methods however represent only one, and potentially limited, means of insight into individuals' experiences. 
% 
The growing commercial success of smart speaker devices following recent advancements in speech recognition technology has surfaced new opportunities for collecting self-reported health and wellbeing data. Speech-enabled conversational agents (CAs) in particular, deployed in home environments using just such systems, may offer increasingly intuitive and engaging means of self-report. To date, however, few real-world studies have examined users' experiences of engaging in the self-report of mental health using such devices, nor the challenges of deploying these systems in the home context.
% 
With these aims in mind, this paper recounts findings from a four-week `in the wild' study during which 20 individuals with depression or bipolar disorder used a speech-enabled CA named `Sofia' to maintain a daily diary log, responding also to the WHO-5 wellbeing scale every two weeks.
% 
Thematic analysis of post-study interviews highlights actions taken by participants to overcome CAs' limitations, diverse personifications of a speech-enabled agent, and surprising degrees of acceptance of this system among users' personal and social circles.
% 
These findings serve as initial evidence for the potential of CAs to support the self-report of mental health and wellbeing, while highlighting the need to address outstanding technical limitations in addition to design challenges of conversational pattern matching, filling unmet interpersonal gaps, and the use of self-report \ac{CA}s in the at-home social context.
% 
Based on these insights, we discuss implications for the future design of CAs to support the self-report of mental health and wellbeing.

% UNUSED NOTES
% often collected in textual or graphical forms using pen and paper or online tools which are 
% What are the findings?
% to understand the potential of such agents and participants' perspectives towards the technology. 
% Health questionnaires and diary logs completed via self-report are considered methods of monitoring and assessing \acp{AD} including depression and bipolar disorder. While these methods are long-established, they represent only one, and potentially limited, means of insight into individuals' experiences. 
% Recent advancements in speech recognition technology and the growing popularity of smart speakers have the potential to enable new opportunities for collecting self-reported health and wellbeing data. \acp{CA} in particular may allow for more natural and engaging means of mental health data collection. To date, however, few real-world studies have examined users' experiences of engaging in such practices, nor the impact of critical design choices. With the aim of understanding the potential of \acp{CA} to support the practice of health self-report, we conducted a four-week `in the wild' study during which $20$ individual users with \ac{AD} used a \ac{CA} named \acl{app} to maintain a daily diary log, while also responding to the \acs{WHO-5} questionnaire every $2$ weeks. 
% %
% Thematic analysis of the interviews shows users' positive perception of the \ac{CA} despite technical limitations, as reflected by their actions taken to overcome the \ac{CA}'s limitations, diverse personification of the agent, social circle's acceptance, and enthusiasm for improving conversation design for improved self-report engagement.
% In addition to the \ac{CA} limitations which prevented users from completing self-reports, primary barriers identified to the engaging and effective \ac{CA} self-report experiences include, users' social situation, and privacy challenges including eavesdropping and data security concerns.
% % 
% Based on these insights, we discuss implications for the future design of \acp{CA} to support mental health and wellbeing.
% Our results indicate users' positive perception of a CA for the self-report of mental health . 
%J: While these methods are long-established, they often suffer from low compliance, lack of engagement, and back-filling from patients.
% Participants' social context and privacy challenges, including eavesdropping and data security concerns, were identified as the primary barriers to engaging and effective CA self-report experiences. 