\section{Limitations}

    This study was conducted during the COVID-19 outbreak. Most participants were affected by different levels of lock-downs, social distancing guidance, and travel restrictions. As a result, many participants mentioned that their mental, physical, and social situations at the time of the study were different than usual, which means that the study results may not translate to this population's more typical context. 
    % 
    Some participants might have engaged more often with \acl{app} as they spent more of their time at home. \textit{``Now with Corona, there is not so much else to do in the evenings\ldots and I live by myself, so it's nice that I can have like a daily talk with \acl{app}''} [P14].
    % 
    % Inconsistency in the pre- and post-study interview methods, as described in Section~\ref{sec:post_study}, could also have induced bias in the interview data.
    As described in Section~\ref{sec:post_study}, participants in this study were interviewed online or in-person. Given the nature of this population group, the method of interview could have affected how they expressed their experiences. P6 for example said; \textit{``I think it's actually really nice we're doing this on a zoom meeting. If we have been like in person, I don't think I would have been so open about it. I would have been a bit more shy, if we had to sit face to face and talk about this''}.
    % 
    All study participants were recruited online and compensated with a Google Nest device for their participation. This may have attracted participants who were more likely to be early adopters of the technology, as suggested in prior literature~\cite{lau2018alexa}.
    % 
    Our study includes a small sample size ($N=20$) from one country and hence may not generalize to other contexts.
    
    
    % ------------------------------------------------------------
    
    %, although we are confident that the insights from this study will facilitate a deeper understanding of the factors affecting \ac{CA} self-report behavior and the perceptions of people with \acl{AD}.
    % 
    %, as stated by P6; \textit{``I think it's actually really nice we're doing this on a zoom meeting. If we have been like in person, I don't think I would have been so open about it. I would have been a bit more shy, if we had to sit face to face and talk about this''}.
    
    % In other circumstances, users' engagement as well as their perceptions of the \ac{CA} could differ.
    % Many participants mentioned that their mental, physical, and social situations at the time of the study was different than usual, which means that the study results may not translate to this population's more typical context. P15 and P18, for example, mentioned that they were recovering from COVID infections during the study and stated that their mental, physical, and social situation differed as a result; \textit{``I got COVID-19, which meant that I had a lot of lung problems, I couldn't breathe sometimes, and I felt tired all the time. And then I had to go back to work, which was very stressful. And then it sort of cracked by stress, which led me to having a depression because I was thrown so wildly from the job market''} [P18].



% UNUSED NOTES
% COVID-19 effects 
% study design
% Recruitment
% effect on self-report
% olnly 4 weeks
% inconsistency in pre post interviews 
% Self-reported AD 
% Participants in the study were not clinically assessed for their self-reported mental health condition.


