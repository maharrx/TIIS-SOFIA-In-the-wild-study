
\section{Discussion}
The experiences of using speech-enabled \ac{CA} for the self-report of mental health and wellbeing shared by this vulnerable participant group provide initial insights into their positive perceptions of the technology.
% 
However, the heterogeneity in \ac{CA}'s impact on their personal and interpersonal contexts in their home settings also suggests possible harms of adopting these emerging technologies for such use cases.
% 
Based on our results, we discuss benefits and potential harms of using \ac{CA}s for the self-report of mental health and wellbeing, ethical and conceptual reflections on their design and use for such practices, and implications for the future design of \ac{CA}s for mental health and wellbeing.

    \subsection{Benefits and Harms of \acl{CA}s for the Self-Report of Mental Health}
        
        \subsubsection{Benefits}
        % Coping with self-report burden
        % efficiency + engagement => positive perception + less burden
        As a primary benefit of using \ac{CA}s for mental health self-report, participants of this study demonstrated high levels of engagement (See Section ~\ref{sec:participant_engagement}), which often reflected broad acceptance of the technology as well as positive perceptions of the \ac{CA}'s potential to support mental health and wellbeing. In particular, speech as an efficient, easy to use and attractive mode of self-report could potentially alleviate reporting burden on the users, which is one of the biggest challenges of the current self-report technologies~\cite{harari2016using, van2017experience, doherty2020design}. 
        
        % +2 
        % Natural medium + friendly voice => better self-report quality + therapeutic experience + feeling heard
        Speech as a more natural form of interaction could help improve the quality of the self-reporting experience as participants reported that it allowed them to express their emotions more freely and spontaneously compared to other means of self-report. Likewise, \ac{CA}'s anthropomorphic characteristics could induce a therapeutic self-reporting experience by responding users with empathy in calming, friendly and caring voice. As such, participants in this study also spoke to feeling heard -- a quality that we believe is fundamental for sustainable long term \ac{CA}-user relationships which could be beneficial for encouraging positive behavior change as reported in prior research(e.g.,~\cite{thieme2015designing, bickmore2005establishing}).
        
        % +3 
        % personification => social companion + overcoming stigma
        Participants' involvement in this study required supporting high levels of trust given the potentially vulnerable nature of the population group, the stigma surrounding mental illness, and the novel nature of the technology itself.
        % 
        Participants' perceptions of \acp{app} as a good listener, a companion, and a tool for emotional venting and self-talking suggest the potential of \ac{CA}s to prove of meaningful value to users and shows
        \ac{CA}'s potential in fulfilling user's deficits in social interaction.  
        % 
        Such connections and associations may also serve as a means of overcoming stigma for people with \ac{AD}s as they often tend to have less social interactions due to the social stigma attached their mental condition as reported by the participants in this study.
        
        % Participants' perceptions of \acp{app} as a good listener, a companion, and a tool for emotional venting and self-talking suggest the potential of \ac{CA}s to prove of meaningful value to users and shows \ac{CA}'s potential in fulfilling user's deficits in social interaction. This could be valuable for people with \ac{AD}s as they often tend to have less social interactions due to the social stigma attached their mental condition.
        

        % \ac{CA} Interventions reduced positive and negative volitional stigma but not traditional stigma (social distance)~\cite{sebastian2017changing}
        
        
        \subsubsection{Potential Harms}
        % -1
        % 12 sec limitation = frustration/stress + expectation gap + adverse effect on mental state
        One of the potential harms of using \ac{CA}s for the self-report of mental health stems from the \ac{CA}'s technical limitation that prevented users from completing their open-ended self-reports. Due to this restriction, participants in this study reported frustrating and disconnected self-reporting experiences despite setting a realistic expectations before \ac{CA} use. Prior studies~\cite{Gulfbetwee2016ewa, onceakind2019minji} have reported similar gap between users' expectation and \ac{CA}'s capacity to engage in a human-to-human-like conversation. As experienced by some participants in this study, such mismatch between users' expectation and \ac{CA}'s capacity could add unwanted stress to their mental state and adversely affect their wellbeing. 
        % 
        Nonetheless, to overcome \ac{CA}'s constraints on the open-ended self-report, many participants employed tactics such as making multiple entries, adopting to \ac{CA}'s speech, and preparing in advance. While the tactic of adopting to \ac{CA}'s speech has been reported in prior research~\cite{myers2018patterns}, making multiple entries and preparing in advance to self-report indicate users' willingness to use the technology despite its limitations. Many participants in this study found these strategies feasible to fully express their emotions and therefore it is important to resist a narrow focus on what the system can do, and to think about how the technology best supports participants' own health and wellbeing goals, and how users are best able to appropriate the technology into their own lives. However, although it was not stated, the requirement of these additional actions to appropriate the technology undoubtedly added a burden on the users, which undermines the usability of these systems for the self-report of mental health and wellbeing.
        
        % -3
        % privacy and security => paranoia/anxiety + bad quality of self-report
        Another potential harm of using \acp{CA} for mental health self-report concerns privacy and data security issues associated with \ac{CA}s. As reported by some participants in the study, their privacy concerns did not only lead to anxiety and paranoia, it also demotivated them to fully express their emotions. However, many participants felt sufficiently able and assertive to take the steps necessary to establish bounds on the use and sharing of their data in line with their own levels of comfort and trust because of the transparency adhered in this study that informed the users before they participated in the study that the transcripts of their self-reports would be accessible to the primary investigator of the study. 
        % 
        Aligned with prior findings on users' privacy seeking behavior while using \ac{CA} (e.g., ~\cite{pradhan2018accessibility, lau2018alexa}), some participants in the study turned off the smart speaker when not in use. Yet others held back on sharing sensitive information on their self-report. Such appropriation of the technology into their lives and for their own purposes despite privacy concerns therefore reflects diverse forms of value associated with the technology.
        % This is particularly important for users living with \ac{AD} and other vulnerable populations.

        % These potential harms of adopting \ac{CA}s for the self-report of mental health and wellbeing suggests the need for improving conversational design to make use of this emerging technology. 
    
    
    % \subsection{Stigma, Vulnerability \& Design Ethics}
    
    %     Participants' involvement in this study required supporting high levels of trust given the potentially vulnerable nature of the population group, the stigma surrounding mental illness, and the novel nature of the technology itself.
    
    %     Participants' perceptions of \acp{app} as a good listener, a companion, and a tool for emotional venting and self-talking suggest the potential of \ac{CA}s to prove of meaningful value to users, and yet also raises questions concerning the potential ethical ramifications of \ac{CA} personification. Such connections and associations may, for example, serve as a means of overcoming stigma, or strengthening it, should that stigma become attached to the technology itself. % \ac{CA} Interventions reduced positive and negative volitional stigma but not traditional stigma (social distance)~\cite{sebastian2017changing}

    %     It is also therefore important to question the extent to which these agents are truly capable of serving as companions, and indeed the extent to which we desire them to do so. The therapeutic, and healthcare literature more broadly, highlights the role of relationship and rapport as mediating factors in the efficacy of care. Can a \ac{CA}, for example, play the role of a therapist? This research suggests that users may find value in connecting to agents to a greater extent than even we as developers imagined given the limited nature of this agent, and yet also underlines the need for future research if these technologies are to be implemented and deployed in ethical ways.


    \subsection{Conversational Self-Report | A Relationship-Oriented Design Framing}
    
        One of the as-of-yet unspoken questions underlining this framing of technology concerns whether self-report is best framed as a conversation at all. And indeed, where the value lies in doing so. The term `report' may be read as implying a power imbalance which did not necessarily reflect how all participants of this study conceived of their relationship to \acl{app}. 

        Indeed, participants' descriptions of \acl{app} itself ranged from `this round little thing' to `this little person in my life;' more often than not leaning in the direction of humanization and personification of the agent. As these devices grow ever more ubiquitous, researchers are beginning to question, as in this work, ways of understanding and designing for increasingly meaningful and longitudinal forms of interaction. This is not a trivial change in orientation, and one which has featured strongly in our thinking around and framing of the design of \acl{app}. As designers and developers of these systems we are increasingly called upon to consider questions pertaining to the kinds of relationships we hope to embody and support via these systems, and the expectations we intend to set for users in this respect. Do we aspire to realize, for example, professional, casual, therapeutic or wholly transactional relationships, and which design choices realize these modes not only of interacting, but relating?
   
   
    \subsection{Desired Futures} % What else needs to be done to improve \ac{CA} design for self-reports of mental health and wellbeing?
    
        \begin{quote}
        \vspace{2mm}
            \textit{``\acl{app} can be someone to talk to\ldots I don't know if you have seen the movie, `Her'~\cite{spike2014her}. Something like that. The `Her' thing is not to automate your life or to remind you anything. Google Assistant can do that. Someone to talk with, not to get any feedback, just to review your day\ldots''} [P16]
        \vspace{2mm}
        \end{quote} 
    
        Looking forward, as smart-speaker devices become more mainstream, it is worth considering the futures we desire, and how we might design to sustainably support them. What role do we want these devices to play in our lives? 
        
        Although most participants expressed hesitation in considering health recommendations provided by a \ac{CA}, several mentioned that they would welcome non-intrusive recommendations that reminded them of the things they could do, although added that a \ac{CA} should grant them broad scope to develop and make their own decisions; \textit{``You can remind sometimes. Like `Do you want to do a breathing exercise to improve your mental health?' and then you can say `Yes' or `No'. I think it's important that you still have the possibility to say no''} [P14]. Other participants suggested incorporating humor and health advice into conversations with \acl{app}.
        
        These possibilities raise additional questions. Do we, for example, want agents to provide recommendations for our health and wellbeing, or prove truly capable of holding rich conversations, or might we prefer uncaring machines, who may also ironically be best placed to provide an experience of care? These are the kinds of questions raised by this study, as one of the first to engage such a participant group in the real-world use of \ac{CA} technologies, and we hope will serve to fuel future research efforts, as we continue to reflect on the implications of growing adoption of these technologies.
    
    
        
    
    % \subsection{Implications for Designing future \acl{CA}s to Support Self-Report of Mental Health and Wellbeing }
    \subsection{Design Implications}
        Our results suggest the need for several considerations to design \ac{CA}s for the self-report of mental health. Reflecting on our findings, we envision future \ac{CA}s for mental health self-report as intelligent agents which can engage users in a wide range of interactions beyond collecting emotional self-reports. 
        % Here, we discuss implications of our findings for designing such \ac{CA}s.
        Here, we discuss potential ways to capitalize the benefits of these systems to support the self-report of mental health and wellbeing.
        
        \subsubsection{Transparency on Privacy and Data Security to Support Trust}
            Results from this study echo prior findings that advocate transparency on \ac{CA}'s privacy policies and data security practices to build trust between the users and \ac{CA}s~\cite{lau2018alexa}. The need for users with mental illness to trust the technology is even more significant as privacy and data security concerns could have adverse effects on their mental health and well-being, as reported in Section~\ref{sec:eavesdropping}. Participants in this study were made aware of the \ac{CA}'s data practices, including what data is collected, who has the access to it and how it is protected. Transparency on such information helped participants trust the system and enabled them to take necessary steps to use the system on their own discretion. It should also be noted that although privacy resignation is becoming more common and nuanced due to the pervasive collection of user data~\cite{lau2018alexa}, transparency on privacy and data practices can foster a positive relationship between \ac{CA} and users, promoting a longitudinal engagement with the system.
            
        \subsubsection{Transparency about \ac{CA} Limitations to Set Users' Expectations}
            Research suggests that due to the conversational nature of the interaction, users often tend to have higher expectations from the \ac{CA}s~\cite{Gulfbetwee2016ewa}. Participants in this study were informed about \acl{app}'s limitations to set users' expectations, which enabled them to appropriate the system by using tactics to overcome the limitations. The disclosure of the \ac{CA} limitations also alleviated otherwise frustrating self-reporting experience by filling the gap between their expectation and the \ac{CA}'s capacity.
        
        \subsubsection{Improving \ac{CA} Design by Varying, Probing and Guiding for Positive Conversation}
            Based on the experience using \ac{CA} for mental health self-report for $28$ days, participants in this study suggested to improve the conversation by varying the questions, probing their responses for more information, and guiding with appropriate response option (See Section~\ref{theme:design_recommendations}). Although these design recommendations may not be unique to this study, the context of mental health self-reports involving potentially vulnerable nature of the population group is worth noting. As such, the participants' suggestion to enable \ac{CA} to guide them to talk about positive things in their lives reflects more of a therapeutic intervention~\cite{tindall2017behavioural} than how it is interpreted in some of the well established heuristics for designing \ac{CA}s~\cite[Table 2. G10]{murad2019revolution}~\cite[Table 8]{langevin2021heuristic}. Likewise, varying questions and probing on the responses also needs to account for the users' mental state to engage them in a positive conversation.
            
        \subsubsection{Supporting (Verbal) Self-reflection}
            Prior research (e.g.,~\cite{kocielnik2018designing, myers2018patterns}) in speech-enabled \ac{CA} has mostly relied on \ac{GUI} to allow users to reflect on their data. While recommendation for accompanying mobile or web app self-reflection was common, participants' suggestion to incorporate verbal reflections of their wellbeing within the self-reporting experience is unique to this study. Participants perceived that the verbal reflection on their data could be more valuable than traditional \ac{GUI}-based tools in supporting health behavior change. Future research could look into designing for such reflection as it entails several challenges including positioning and automated vs. on demand provision of the reflection as discussed by the participants in this study (Section~\ref{sec:reflection}).
        
        \subsubsection{Imbuing Conversational Characteristics to Support Social Goals}
            Despite the understanding that the \ac{CA} is just a machine, participants' narratives in this study, including personification of the \ac{CA} in various forms indicate the benefits for imbuing conversational characteristics in \ac{CA} design to support users' unmet gap in social interactions. While the participants did not expect \ac{CA} to understand their emotions, they still wanted the illusion from the \ac{CA} that it cared, indicating the need for \ac{CA}'s sensitivity to the context of interaction including the sentiment of the users' utterance, topic of the conversation and ability to formulate relevant follow up questions based on a user response.
        
        \subsubsection{Imbuing Conversational Characteristics to Support Empathy}
            Consistent with the prior findings, our results show that participants valued \acl{CA} as a `good listener'~\cite{bauer2010introducing, clark2019makes}. Although not intended, participants' in this study appreciated \ac{CA}'s compassionate feedback which gave them a sense of being heard. Their desire for emotional support from the \ac{CA} suggests the need for designing agents with ability to formulate empathetic responses to user utterances to improve engagement. With the possibility of tailoring voice features such as tone, intonation, speed and pitch, speech-enabled \ac{CA}s could emulate an empathetic self-reporting experience which could have positive effects for their mental health and wellbeing as demonstrated by
            prior work on text-based~\cite{inkster2018empathy}. 


    % \subsection{Enablers and Barriers to User Engagement}
    
    %     Engagement is a prominent goal for the design of many systems within \ac{HCI}, and has been touted as one of the potential advantages of speech-enabled systems. One way to support design for engagement in the practice of self-report via \ac{CA} is therefore to begin to understand its enablers and barriers.
        
    %     % Enabler:convenience
    %     Many participants of this study demonstrated high levels of engagement (See Sections \ref{sec:participant_engagement},~\ref{sec:perceived_experience}), which often reflected broad acceptance of the technology as well as positive perceptions of the \ac{CA}'s potential to support mental health and wellbeing. 
    %     % As motivations for their engagement in self-report via an agent, many participants, in line with prior findings~\cite{lau2018alexa, Rafal2018Workplace}, pointed to the convenience offered by the \ac{CA}'s hands-free experience; \textit{``I love it because it’s like an interface you talk directly to. It’s super easy to use. You don’t have to open your laptop and go to a specific page. I can just go home, open the door and talk to \acl{app}, super easy''} [P1]. 
        
    %     % Enabler:natural interaction
    %     % Speech, as a more natural form of interaction, was also described as allowing users to express their emotions more freely and spontaneously compared to other means of self-report; \textit{``Speaking is much easier because you can just let the words flow and you don't have to think about it''} [P14].
        
    %     Still other participants spoke to feeling heard -- a quality that we believe is fundamental for sustainable long term \ac{CA}-user relationships. Users' appropriation of the technology into their lives and for their own purposes therefore reflects diverse forms of value associated with the technology. P14 and P6, for example, even took the smart speaker with them when away from home; \textit{``So for three weeks I was in the sort of caravan and actually it worked fine. I just had to reset it to my, to the Wi-Fi, and that was not a problem''}[P14].       
            
    %     % Barriers:technical limitations which prevented users from completing their self-reports, social factors, and privacy concerns including eavesdropping and data security
    %     And yet, as barriers to their engagement, participants often mentioned technical limitations which prevented users from completing their self-reports, social factors, and privacy concerns including eavesdropping and data security. Our findings therefore reflect participants' willingness to engage with \acp{CA} for the self-report of mental health and wellbeing, but also raise questions about the readiness of \acp{CA} for real-world use.
 
    %     When thinking about engagement as a design goal it is therefore important to resist a narrow focus on engagement as an aim in itself, and to think about how the technology best supports participants' own health and wellbeing goals, and how users are best able to appropriate the technology into their own lives. This is particularly important for users living with \ac{AD} and other vulnerable populations.
    
 
    
    
    
    
            

% UNUSED NOTES
% \textbf{Our findings reveal users' experience as strongly shaped by actions taken to overcome the technological limitations of \acp{CA}, diverse personified perceptions of a self-report agent, the socially-contingent nature of self-reporting practice as well as users' reflections on privacy and security concerns.}
% \subsubsection{Emulate emotional support}
% \subsubsection{Establish positive tone or relationship}
% Privacy perception and data security concerns presented in this study show users' trust towards the governing bodies, yet their confidence over smart speaker companies remain skeptical. 
% Based on these results, we discuss: a, b, c 
% with respect to \ac{CA}'s potential in empowering users to self-report their mental health and wellbeing.
% We then, highlight several important design considerations for future design of \ac{CA} to support self-report of mental health and wellbeing.
% Notably, this result also reflects the gulf between the technology and the users expectation of being able to have a human-human-like conversation~\cite{Gulfbetwee2016ewa} despite setting a realistic expectations before \ac{CA} use. This leads 
% the question whether the technology is capable of engaging users for a desired outcome.
% participants with lower than 60\% of adherence rate (P1, P11, P20) reasoned that they were unable to use \ac{CA} due to the immobility of the smartspeaker, their own mental state and privacy reasons. 
% % immobility of the smartspeaker, 
% \textit{``When I am quite down, I close down. I don’t talk to anyone. It’s not just \acl{app}. During the times I was lived with my boyfriend, it was difficult for me to ask him to leave the room as I wanted privacy to talk to \acl{app}. And sometime, I wasn’t in town''}.
% Enablers
% 1. There are so many technical limitations of \ac{CA} but people are ready and willing to use the system - reflected by our results:
% UEQ 
% Adherence
% tactics they used and 
% the social and mental benifits they outlined.
% Enablers of Engagement
% Despite many technical challenges, results 
% Will \ac{CA} be able to relate 
% P14 in particular questioned a \ac{CA}'s ability to relate users' emotions to their past and commented as a result that \acp{CA} might not be able to establish truly effective patient-therapist (\ac{CA} or human) relationships while acknowledging that talking to \acp{CA} might help one to reflect on their emotions.
% \textit{``You can have self reflection and you can figure out things by yourself. But sometimes it is also important that someone else can say `okay, you're feeling this way but maybe it comes from this' and they can relate things that have been mentioned in other conversations or those relationships cannot be made with \acl{app} so and that's why I think it's more useful to have sometimes meetings with the psychiatrists.''} [P14]
% Reasons for Low Rates of
% Participants with lower than 60\% of adherence rate (P1, P11, P20) reasoned that they were unable to use \ac{CA} due to
% the immobility of the smartspeaker, 
% their own mental state and 
% privacy reasons. 
% immobility of the smartspeaker, 
% P1, for example, when asked about her low adherence rate, they explained: 
% \textit{``When I am quite down, I close down. I don’t talk to anyone. It’s not just \acl{app}. During the times I was lived with my boyfriend, it was difficult for me to ask him to leave the room as I wanted privacy to talk to \acl{app}. And sometime, I wasn’t in town''}.
% Despite the static nature of the smart speaker, P14 and P6 however, took the smart speaker with them when they were away from their home and described that it was not much of a hassle to set up the system in a new place as long as they had access to the electric outlet and Wi-fi.
% \begin{quote}
% \textit{``So for three weeks I was in the sort of caravan and actually it worked fine. I just had to reset it to my, to the Wi-Fi, and that was not a problem.''}
% [P14]
% r27
% \end{quote}    
% Participants shared many valuable suggestions for the future design of a \acp{CA} to support self-reports of mental health and wellbeing. Their suggestions included ways to 
% (i) improve \ac{CA}'s conversational skill;
% (ii) reflect on self-reports; and provide
% (iii) health recommendations.
% ``past is just a story we tell to ourselves''~\cite{spike2014her}
% Who are they reporting to in participant's view?
% Who is the conversation with?
% Where and how does `conversation' add nuance?
% P7 shared similar thoughts and added that appropriate probing would make the conversation more natural and emulate the sense of being heard.
%     \textit{``I would like some maybe maybe basic minimal but some sort of a simulation of listening you know, like she like she would pick up on the small things in the conversation and ask an ask questions based on that. Of course that's the hardest part you know, making them think Yeah, you know, like the way the natural people communicate you know, like if someone if someone said you know, `Man my boss really frustrates me', you know, then \acl{app} could ask him, `Oh, what is so frustrating about her?' or `What did she say?' and stuff like that. That sort of things will make the conversation more organic''}.
% ----sense of confrontation----
% ---- \ac{CA} doesn't make them feel like shit ----
% HOW to present the data for reflection?
% WHY NOT Vocal reflection?
% It is therefore important to ask the question as to whether \ac{CA} technology is ready for such use cases. 
% Suggesting a supplemental mobile app to reflect on their data, P7 said, 
% \textit{``If \acl{app} was telling me my statistics, I would probably space out through half of it. And I would be like, yeah, is there an option to repeat this? So a supplemental app would be great''}.
% Assuming that \acl{CA} would play the recordings of their self-reports to reflect on P5 noted that they did not like to listen to themselves and the graphical would be more appropriate for reflection:
% \textit{``I don’t like to listen to myself. Reading could be better. You can search easily with your eyes and go through the words - find interesting parts''}.
% Participants shared a common view of the need for a visual tool to reflect on their data although many also entertained the idea of verbal reflection. While several participants expressed their interest in verbal reflection, some also believed that it could serve as an effective tool for sustainable behavior change.
% This included (i) making multiple entries, (ii) adapting one's speech, and (iii) preparing in advance to self-report. A
% Many of participants comments also reflected their own respect for the thoughts of their social circle.
% Narratives of the participants' social circle strengthens \ac{CA}'s social acceptance and its potential in supporting users' relationship with their significant others, albeit, .
% Awareness of this position raises questions concerning the ethics of various framings of the technology. Many researches are working on personified interpretations of agents [cite work on penguins etc], and in this study many participants assigned human-like qualities to \acl{app}. And yet 
% Participants suggested several other suggestion to support engaging self-report via \ac{CA}. These design suggestions included 
% (i) notification, 
% (ii) humor, and
% (iii) health information.
% notification for adherence
% P1 and P16 suggested notifying users to talk to \ac{CA} if they have not self-reported for certain period of time.
% \textit{``When I am down, I close myself to the world, it would be an important moment for me to actually speak. Therefore, if I have not been speaking for lets say 48 hrs, \acl{app} could notify me or ask me, `Hey, are you feeling okay?' ''} [P1].
% humor
% Echoing the prior literature~\cite{clark2019makes}, P4 remarked humor in the conversation could make them feel positive:
% \textit{``If you can get people to smile maybe they'll be more positive about their day...say something funny''}.
% information on the issue	
% P4 also mentioned that providing relevant health information (e.g., symptoms of a health condition) could make feel that they are not the alone:



% \textit{``It could read you something from the web page, like `(name of the site) says this about anxiety or blah blah blah'. That makes you kind of like, Okay other people feel these things also when they have anxiety'' }.
% \subsection{Users Are Ready to Engage. How about \acp{CA}?} % Is \ac{CA} technology ready?
%     Engagement is a prominent goal for the design of many systems within \ac{HCI}, and has been touted as one of the potential advantages of speech-enabled systems. How then might we design for engagement in the practice of self-report via \ac{CA}?


%     \subsubsection{To use a \ac{CA} might be convenient but having conversation with it is a different story.} 
%         % Its is convenient 
%         Many participants of this study demonstrated high levels of engagement (See Sections \ref{sec:perceived_engagement} \& ~\ref{sec:perceived_experience}) which often reflected broad acceptance of the technology as well as positive perceptions of the potential of \acp{CA} to support mental health and wellbeing. As motivations for their engagement in self-report via an agent, many participants, in line with prior findings~\cite{lau2018alexa, Rafal2018Workplace}, pointed to the convenience offered by the \ac{CA}'s hands-free experience; \textit{``I love it because it’s like an interface you talk directly to. It’s super easy to use. You don’t have to open your laptop and go to a specific page. I can just go home, open the door and talk to \acl{app}, super easy''} [P1]. 

%     % But doesn't have the fundamental conversational feature like pause 
%     % More research on designing c?



%     \subsubsection{Being heard is important for this group of population but how can we design for it?} 
%     % how can CA do that? Is it capable?
%         Speech as a more natural form of interaction, allows users to express their emotions freely and spontaneously compared to other means of self-report; \textit{``Speaking is much easier because you can just let the words flow and you don't have to think about it''} [P14]. Still others spoke to feeling heard -- a quality that we believe is fundamental for sustainable long term \ac{CA}-user relationships. And yet, as barriers to their engagement, participants often mentioned technical limitations which prevented users from completing their self-reports, social factors, and privacy concerns including eavesdropping and data security. Our findings therefore reflect participants' willingness to engage with \acp{CA} for the self-report of mental health and wellbeing, but also raise questions about the readiness of \acp{CA} for real-world use. 

%     \subsubsection{Does CA has to be that sophisticated? What do users want?} 
%         Technical limitations (See Section~\ref{sec:ca_limitations}) undoubtedly restrained users from utilizing the \ac{CA} to its full potential, a result that aligns with prior research findings~\cite{Rafal2018Workplace}. Interestingly many of the limitations which most challenged participants are not strictly technical in nature but due to artificial constraints imposed on the interaction design itself. This highlights the need for a future human-centered approach to \ac{CA} design and broader consideration of a wider variety of use-cases for such systems. 

%         % participants were 
%         \begin{quote}
%         \vspace{2mm}
%             \textit{``If \acl{app} can do, not to automate the life, to be someone to talk to. Yes, I can do that everyday. I can trust to talk to \acl{app} everyday. I don't know if you have ever seen the movie, `Her'~\cite{spike2014her}. Something like that. The `Her' thing is not to automate your life or to remind you anything. Google Assistant can do that. Someone to talk with, not to talk with to get your feedback, just to talk with to review your day and to see okay what you're actually doing.''} [P16]
%         \vspace{2mm}
%         \end{quote} 

%         Looking forward, as smart-speaker devices become more mainstream, it is worth considering the futures we desire, and how we might design to sustainably support them. What role do we want these devices to play in our lives? 

%         Although most participants expressed hesitation in considering health recommendations provided by a \ac{CA}, several mentioned that they would welcome non-intrusive recommendations that reminded them of the things they could do, although added that a \ac{CA} should allow them to make their own decisions to act;  \textit{``I mean sometimes you forget. So you can remind sometimes. It should be not like `Okay, here you have a breathing exercise'. Like `Do you want to do a breathing exercise to improve your mental health?' and then you can say `Yes' or `No'. I think it's important that you still have the possibility to say no''} [P14]. Other participants suggested incorporating humour and advice into conversations with \acl{app}.

%         These possibilities raise additional questions. Do we, for example, want agents to provide recommendations for our health and wellbeing, or prove truly capable of holding rich conversations, or might we prefer uncaring machines, who may also ironically be best placed to provide an experience of care? These are the kinds of questions raised by this study, which is one of the first to engage a vulnerable group in the real-world use of such technologies, despite the fact that many such systems are available and continue to emerge on the commercial app stores. It is time we begin to reflect on the implications of growing adoption of these technologies, and we hope this work serves as a step in the direction of such a discourse.      
% This study therefore reveals the extent to which, when designing for engagement, it is also important to consider users', designers' and researchers' implicit and explicit motivations. 
% Is there such a thing as `the uncanny value of caring'? Where does it start and where does it end?
% The phrase `designing for conversation' may therefore itself be read as a re-framing of the very idea of self-report.
% Indeed, even more broadly, which characteristics of human-to-human relationships translate to human-to-agent relationships?
% In part, participants' varied interpretations of the agent may be explained by our own open-ended presentation of the system -- a conscious choice. We often struggled to provide a concise description of our own framing of this agent, something we have noticed much nascent work in this space struggles to achieve, as researchers and designers strive to strike the appropriate balance between human and machine. To overcome such barriers to action and clarify our own expression, it can be useful to ask ourselves such questions as `From the participant's point of view, who, or what, are they reporting to?' Who is the conversation with? And where does conversation add value? Making these kinds of conceptions clear to participants is likely key to engendering trust.

% \ac{VUI} technologies are increasingly presented as capable of supporting more human modes of interaction, and in turn characterized and analyzed in terms of interactional traits, from responsiveness to tone and ease of understanding. We argue that designers of \ac{VUI} systems must also consider how particular relationship framings interact with the limitations and possibilities of technology. A more professional framing for example, can enable designers to side-step certain limitations of the technology by, from the offset, setting expectations which preclude more casual and intimate forms of interaction. We must then ask, which technological futures do we desire?