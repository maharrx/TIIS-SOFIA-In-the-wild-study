 % \begin{table}[]
    \small
    \centering
    \caption{Challenges to the self-report of mental health and wellbeing via \ac{CA} and design implications}
    \begin{tabular}{p{4cm}p{9.25cm}}
        \toprule
        \textbf{Challenges} & \textbf{Design Implications}\\
        \midrule
            
            \begin{itemize}[leftmargin=0em]
                \item[]  The Challenge of Conversational Pattern Matching
            \end{itemize}   
            
        & 
             
            \begin{itemize}[leftmargin=1em]
                \item  Tailor opportunities for self-expression through continuous probing and guiding of the conversation.
            \end{itemize}\\

             
            \begin{itemize}[leftmargin=0em]
                \item[] The Challenge of Filling the Right Gap
            \end{itemize}   
        & 
             
            \begin{itemize}[leftmargin=1em]
                \item Attend to and provide space for human experiences of sociality, connectedness, empathy and compassion, while allowing users to appropriate technology in the ways they see fit.
            \end{itemize}\\
            
             
            \begin{itemize}[leftmargin=0em]
                \item[] The Challenge of The At-Home Social Context
            \end{itemize}   
        & 
             
            \begin{itemize}[leftmargin=1em]
                \item Suggest transparent communication of \ac{CA} privacy policies and data practices in addition to educating users about privacy settings, in order to build trust between users and these systems.
            \end{itemize}\\
        \bottomrule
    \end{tabular}
    \label{tab:challenges}
\end{table} 

    % Theme 1 : Risk 1 
    \subsection{Unintended Reporting Burden}
        
        Participants in this study demonstrated high levels of engagement (See Section ~\ref{sec:participant_engagement}), and described \acl{app} as efficient, easy to use and attractive medium of mental health self-report. While these qualities of \ac{CA}s could potentially alleviate reporting burden on the users --- one of the biggest challenges of the current self-report technologies~\cite{harari2016using, van2017experience, doherty2020design} --- its technical limitation that prevented users from completing their open-ended self-reports. To overcome such constraints, many participants employed tactics such as making multiple entries, adopting to \ac{CA}'s speech, and preparing in advance indicating users' willingness to use the technology despite its limitations (Section~\ref{sec:strategies_to_overcome}). While many participants found these strategies feasible and applied them to express their emotions fully, the requirement of these additional actions to appropriate the technology undoubtedly added a burden on the users resulting in frustrating and disconnected self-reporting experiences. 
        
        % Solution
        % One way of mitigating such potential burdens could be to improve conversation design by allowing continuous interaction. Our \ac{CA} design asked only three open-ended questions to respond about their mental health. Future \ac{CA} designs could implement multiple \textit{follow-up} questions to enable users express their emotions fully.
        

    % Risk 2
    \subsection{Stigma}
        Participants' involvement in this study required supporting high levels of trust given the potentially vulnerable nature of the population group, the stigma surrounding mental illness, and the novel nature of the technology itself. Participants' perceptions of \acp{app} as a good listener, a companion, and a tool for emotional venting and self-talking suggest the potential of \ac{CA}s to prove of meaningful value to users and shows \ac{CA}'s potential in fulfilling user's unmet gap in social interactions. Such connections and associations may also serve as a means of overcoming stigma for people with \ac{AD}s as they often tend to have less social interactions due to the social stigma attached their mental condition as reported by the participants in this study. Participants' descriptions of \acl{app} in this study ranged from `this round little thing' to `this little person in my life;' more often than not leaning in the direction of humanization and personification of the agent. Some expressed that they would trust \ac{CA}s more than their human counterparts as the \ac{CA} made them feel heard. While these associations create opportunities for building therapeutic relationships between \ac{CA} and the users, it also raises questions concerning the potential ethical ramifications. As these devices grow ever more ubiquitous and technologically advanced to engage users in social conversations, there is potential for this vulnerable population group to become over-attached and even dependent to these systems, further isolating them from their social circle and distancing them from their personal relationships. In turn, the technology itself could be stigmatized. 
        
        

    
        